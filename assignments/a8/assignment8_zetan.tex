%%%%%%%%%%%%%%%%%%%%%%%%%%%%%%%%%%%%%%%%%
% Programming/Coding Assignment
% LaTeX Template
%
% This template has been downloaded from:
% http://www.latextemplates.com
%
% Original author:
% Ted Pavlic (http://www.tedpavlic.com)
%
% Note:
% The \lipsum[#] commands throughout this template generate dummy text
% to fill the template out. These commands should all be removed when 
% writing assignment content.
%
% This template uses a Perl script as an example snippet of code, most other
% languages are also usable. Configure them in the "CODE INCLUSION 
% CONFIGURATION" section.
%
%Assignment 6
%Author Zetan
%%%%%%%%%%%%%%%%%%%%%%%%%%%%%%%%%%%%%%%%%

%----------------------------------------------------------------------------------------
%	PACKAGES AND OTHER DOCUMENT CONFIGURATIONS
%----------------------------------------------------------------------------------------

\documentclass{article}

\usepackage{fancyhdr} % Required for custom headers
\usepackage{lastpage} % Required to determine the last page for the footer
\usepackage{extramarks} % Required for headers and footers
\usepackage[usenames,dvipsnames]{color} % Required for custom colors
\usepackage{graphicx} % Required to insert images
\usepackage{listings} % Required for insertion of code
\usepackage{courier} % Required for the courier font
\usepackage{multirow}
\usepackage{listings,multicol}
\usepackage{pgfplots,pgfplotstable}
 \usepackage{amssymb}
 \usepackage{float}

\usepackage{url}

% Margins
\topmargin=-0.45in
\evensidemargin=0in
\oddsidemargin=0in
\textwidth=6.5in
\textheight=9.0in
\headsep=0.25in

\linespread{1.1} % Line spacing

% Set up the header and footer
\pagestyle{fancy}
\lhead{\hmwkAuthorName} % Top left header
\chead{\hmwkClass\ (\hmwkClassInstructor\ \hmwkClassTime): \hmwkTitle} % Top center head
\rhead{\firstxmark} % Top right header
\lfoot{\lastxmark} % Bottom left footer
\cfoot{} % Bottom center footer
\rfoot{Page\ \thepage\ of\ \protect\pageref{LastPage}} % Bottom right footer
\renewcommand\headrulewidth{0.4pt} % Size of the header rule
\renewcommand\footrulewidth{0.4pt} % Size of the footer rule

\setlength\parindent{0pt} % Removes all indentation from paragraphs

%----------------------------------------------------------------------------------------
%	CODE INCLUSION CONFIGURATION
%----------------------------------------------------------------------------------------
\definecolor{lightgray}{rgb}{.9,.9,.9}
\definecolor{darkgray}{rgb}{.4,.4,.4}
\definecolor{purple}{rgb}{0.65, 0.12, 0.82}
\definecolor{MyDarkGreen}{rgb}{0.0,0.4,0.0} % This is the color used for comments
\lstloadlanguages{Python} % Load python syntax for listings, for a list of other languagesftp://ftp.tex.ac.uk/tex-archive/macros/latex/contrib/listings/listings.pdf supported see: 
\lstdefinelanguage{JavaScript}{
  keywords={break, case, catch, continue, debugger, default, delete, do, else, false, finally, for, function, if, in, instanceof, new, null, return, switch, this, throw, true, try, typeof, var, void, while, with},
  morecomment=[l]{//},
  morecomment=[s]{/*}{*/},
  morestring=[b]',
  morestring=[b]",
  ndkeywords={class, export, boolean, throw, implements, import, this},
  keywordstyle=\color{blue}\bfseries,
  ndkeywordstyle=\color{darkgray}\bfseries,
  identifierstyle=\color{black},
  commentstyle=\color{purple}\ttfamily,
  stringstyle=\color{red}\ttfamily,
  sensitive=true
}
\lstset{
        frame=single, % Single frame around code
        basicstyle=\small\ttfamily, % Use small true type font
        keywordstyle=[1]\color{Blue}\bf, % python functions bold and blue
        keywordstyle=[2]\color{Purple}, % python function arguments purple
        keywordstyle=[3]\color{Blue}\underbar, % Custom functions underlined and blue
        identifierstyle=, % Nothing special about identifiers                                         
        commentstyle=\usefont{T1}{pcr}{m}{sl}\color{MyDarkGreen}\small, % Comments small dark green courier font
        stringstyle=\color{Purple}, % Strings are purple
        showstringspaces=false, % Don't put marks in string spaces
        tabsize=5, % 5 spaces per tab
        breaklines=true,
        %
        % Put standard python functions not included in the default language here
        morekeywords={rand},
        %
        % Put python function parameters here
        morekeywords=[2]{on, off, interp},
        %
        % Put user defined functions here
        morekeywords=[3]{test},
       	%
        morecomment=[l][\color{Blue}]{...}, % Line continuation (...) like blue comment
        numbers=left, % Line numbers on left
        firstnumber=1, % Line numbers start with line 1
        numberstyle=\tiny\color{Blue}, % Line numbers are blue and small
        stepnumber=5 % Line numbers go in steps of 5
}

% Creates a new command to include a pyton script, the first parameter is the filename of the script (without .py), the second parameter is the caption
\newcommand{\pythonscript}[2]{
\begin{itemize}
\item[]\lstinputlisting[language=python,caption=#2,label=#1]{#1.py}
\end{itemize}
}
% Creates a new command to include a shell script, the first parameter is the filename of the script (without .sh), the second parameter is the caption
\newcommand{\shellscript}[2]{
\begin{itemize}
\item[]\lstinputlisting[language=bash,caption=#2,label=#1]{#1.sh}
\end{itemize}
}
% Creates a new command to include a R script, the first parameter is the filename of the script (without .R), the second parameter is the caption
\newcommand{\Rscript}[2]{
\begin{itemize}
\item[]\lstinputlisting[language=R,caption=#2,label=#1]{#1.R}
\end{itemize}
}
% Creates a new command to include a java script, the first parameter is the filename of the script (without .R), the second parameter is the caption
\newcommand{\jsscript}[2]{
\begin{itemize}
\item[]\lstinputlisting[language=JavaScript,caption=#2,label=#1]{#1.js}
\end{itemize}
}
%----------------------------------------------------------------------------------------
%	DOCUMENT STRUCTURE COMMANDS
%	Skip this unless you know what you're doing
%----------------------------------------------------------------------------------------

% Header and footer for when a page split occurs within a problem environment
\newcommand{\enterProblemHeader}[1]{
\nobreak\extramarks{#1}{#1 continued on next page\ldots}\nobreak
\nobreak\extramarks{#1 (continued)}{#1 continued on next page\ldots}\nobreak
}

% Header and footer for when a page split occurs between problem environments
\newcommand{\exitProblemHeader}[1]{
\nobreak\extramarks{#1 (continued)}{#1 continued on next page\ldots}\nobreak
\nobreak\extramarks{#1}{}\nobreak
}

\setcounter{secnumdepth}{0} % Removes default section numbers
\newcounter{homeworkProblemCounter} % Creates a counter to keep track of the number of problems

\newcommand{\homeworkProblemName}{}
\newenvironment{homeworkProblem}[1][Problem \arabic{homeworkProblemCounter}]{ % Makes a new environment called homeworkProblem which takes 1 argument (custom name) but the default is "Problem #"
\stepcounter{homeworkProblemCounter} % Increase counter for number of problems
\renewcommand{\homeworkProblemName}{#1} % Assign \homeworkProblemName the name of the problem
\section{\homeworkProblemName} % Make a section in the document with the custom problem count
\enterProblemHeader{\homeworkProblemName} % Header and footer within the environment
}{
\exitProblemHeader{\homeworkProblemName} % Header and footer after the environment
}

\newcommand{\problemAnswer}[1]{ % Defines the problem answer command with the content as the only argument
\noindent\framebox[\columnwidth][c]{\begin{minipage}{0.98\columnwidth}#1\end{minipage}} % Makes the box around the problem answer and puts the content inside
}

\newcommand{\homeworkSectionName}{}
\newenvironment{homeworkSection}[1]{ % New environment for sections within homework problems, takes 1 argument - the name of the section
\renewcommand{\homeworkSectionName}{#1} % Assign \homeworkSectionName to the name of the section from the environment argument
\subsection{\homeworkSectionName} % Make a subsection with the custom name of the subsection
\enterProblemHeader{\homeworkProblemName\ [\homeworkSectionName]} % Header and footer within the environment
}{
\enterProblemHeader{\homeworkProblemName} % Header and footer after the environment
}

%----------------------------------------------------------------------------------------
%	NAME AND CLASS SECTION
%----------------------------------------------------------------------------------------

\newcommand{\hmwkTitle}{Assignment\ \#8} % Assignment title
\newcommand{\hmwkDueDate}{Thursday,\ April\ 8,\ 2016} % Due date
\newcommand{\hmwkClass}{Web Science\ cs532} % Course/class
\newcommand{\hmwkClassTime}{4:20pm} % Class/lecture time
\newcommand{\hmwkClassInstructor}{Dr.Michael.L.Nelson} % Teacher/lecturer
\newcommand{\hmwkAuthorName}{Zetan Li} % Your name

%----------------------------------------------------------------------------------------
%	TITLE PAGE
%----------------------------------------------------------------------------------------

\title{
\vspace{2in}
\textmd{\textbf{\hmwkClass:\ \hmwkTitle}}\\
\normalsize\vspace{0.1in}\small{Due\ on\ \hmwkDueDate}\\
\vspace{0.1in}\large{\textit{\hmwkClassInstructor\ \hmwkClassTime}}
\vspace{3in}
}

\author{\textbf{\hmwkAuthorName}}
\date{} % Insert date here if you want it to appear below your name

%----------------------------------------------------------------------------------------

\begin{document}

\maketitle

%----------------------------------------------------------------------------------------
%	TABLE OF CONTENTS
%----------------------------------------------------------------------------------------

%\setcounter{tocdepth}{1} % Uncomment this line if you don't want subsections listed in the ToC

\newpage
\tableofcontents
\newpage

%----------------------------------------------------------------------------------------
%	PROBLEM 1
%----------------------------------------------------------------------------------------

% To have just one problem per page, simply put a \clearpage after each problem

\begin{homeworkProblem}
Create a blog-term matrix.  Start by grabbing 100 blogs; include:\\
\\
\url{http://f-measure.blogspot.com/}\\
\url{http://ws-dl.blogspot.com/}\\
\\
and grab 98 more as per the method shown in class.  Note that this
method randomly chooses blogs and each student will separately do
this process, so it is unlikely that these 98 blogs will be shared
among students.  In other words, no sharing of blog data.  Upload
to github your code for grabbing the blogs and provide a list of
blog URIs, both in the report and in github..\\
\\
Use the blog title as the identifier for each blog (and row of the
matrix).  Use the terms from every item/title (RSS) or entry/title
(Atom) for the columns of the matrix.  The values are the frequency
of occurrence.  Essentially you are replicating the format of the
``blogdata.txt" file included with the PCI book code.  Limit the
number of terms to the most ``popular" (i.e., frequent) 500 terms,
this is *after* the criteria on p. 32 (slide 7) has been satisfied.\\
\centerline{SOLUTION}
\textbf{Grab the blogs}\\
To parse the entries' title, we have first get feed uri from raw blog page. According to slides, we can use curl to grab random blog page from uri \\\textbf{\url{https://www.blogger.com/next-blog?navBar=true&blogID=3471633091411211117}}\\
\\
The option -L enable curl to auto-tracking the redirection, and -w option enable curl to print out the final effective uri of the blog page.\\
Ideally, we can get 100 different blogs when we visit this uri 100 times. However, in fact, we get bunch of duplicated uris from it after running query for many times. So we have to make the uris unique. The -u option in sort can fulfill this task.\\
(The raw blog page is not uploaded, only uri list is uploaded to github. In the sub-directory rawBlogs)
\shellscript{p1_grabBlog}{Shell script to download 100 different blog page from internet}
\textbf{Get feed uri}\\
Now with 100 blog pages, we can retrieve rss feed uri from them by BeautifulSoup. According to slides, the rss link is in the link label with type ``application/rss+xml''
\pythonscript{p1_getRSS}{Python code to retrieve the feed uri list}
\textbf{Generate matrix}\\
The code on PCI book \cite{pci} can generate the matrix for us, the only modification on the code is to add a limit to the column we pick.(Notice the code on line 52)
\pythonscript{p1_feeder}{Python code to generate matrix}
For some blog title that contains non-ascii code, we have to encode them in utf-8(code line 60)\\
The matrix is stored in file ``blogdata.txt''.
\end{homeworkProblem}

%----------------------------------------------------------------------------------------
%	PROBLEM 2
%----------------------------------------------------------------------------------------
\begin{homeworkProblem}
Create an ASCII and JPEG dendrogram that clusters (i.e., HAC)
the most similar blogs (see slides 12 \& 13).  Include the JPEG in
your report and upload the ascii file to github (it will be too
unwieldy for inclusion in the report).\\
\centerline{SOLUTION}
According to slides, we can pick hcluster function from PCI to finish this task.\\
\begin{itemize}
\item[]\lstinputlisting[language=python,caption=Python code to plot dendrogram,label=p2,firstline=169,lastline=172]{p2.py}
\end{itemize}
We running this code in the way like\\ \textbf{\url{python p2.py > p2_ascii.txt}}\\
so we can get ascii output into a file instead of printing them on console.\\
The complete code is in p2.py .\\
The ascii version of dendrogram is included in p2\_ascii.txt
\begin{figure}[H]
\centering
\caption{Dendrogram of blogs(large image, check p2\_dendrogram on github)}
\includegraphics[width=5.3in]{p2_dendrogram.jpg}
\end{figure}

\end{homeworkProblem}
\pagebreak
%----------------------------------------------------------------------------------------
%	PROBLEM 3
%----------------------------------------------------------------------------------------
\begin{homeworkProblem}
Cluster the blogs using K-Means, using k=5,10,20. (see slide
18).  Print the values in each centroid, for each value of k.  How
many interations were required for each value of k?\\
\centerline{SOLUTION}
The function kcluster in the PCI book can do this work. The only modification we have to do is to return the iteration time in the function then print it out.
\pythonscript{p3}{Python code to get k clusters}
For output, check p3\_k5.txt for k=5, p3\_k10.txt for k=10 and p3\_k20.txt for k=20 in the github.
\end{homeworkProblem}
\pagebreak
%----------------------------------------------------------------------------------------
%	PROBLEM 4
%----------------------------------------------------------------------------------------
\begin{homeworkProblem}
Use MDS to create a JPEG of the blogs similar to slide 29.  
How many iterations were required?\\
\centerline{SOLUTION}
The function scaledown in the PCI book can finish this task.\\
Here, we let the function return the iteration time at the same time.
\pythonscript{p4}{Python code to get MDS}
\begin{figure}[h]
\centering
\caption{Processing output}
\includegraphics[width=5in]{MDS_out.jpg}
\end{figure}
\begin{figure}[H]
\centering
\caption{MDS(large image, check p4\_MDS.jpg on github)}
\includegraphics[width=7.3in]{p4_MDS.jpg}
\end{figure}
\end{homeworkProblem}
\bibliographystyle{plain}
\bibliography{ref}
\end{document}